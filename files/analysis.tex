\documentclass[../Main.tex]{subfiles}
\begin{document}
\chapter{Sorting Algorithm}

\intro{

}

\section{Ableitungen}
\subsection{Definition}
Definition der Ableitungsfunktion für \(\forall x \in D\) glatt.
\begin{equation}
    f^{'}(n) =
  \begin{cases}
    D       &\rightarrow Z\\
    x_0       &\mapsto \left.\frac{d f(x)}{dx}\right|_{x=x_0}
  \end{cases}
\end{equation}
Ergibt sich aus dem Differenzenquotient:
\begin{equation}
    m_{x_0}(x) = \frac{f(x)-f(x_0)}{x-x_0} 
\end{equation}
Tangentenfunktion.
\begin{equation}
    t_{x_0} = f^{'}(x_0)(x-x_0)+f(x_0)
\end{equation}
\subsection{Ableitungsregeln}
\begin{equation}
    \begin{aligned}
        &\frac{d}{dx}x^a         &= &a \cdot x^{a-1} \\
        &\frac{d}{dx}e^x         &= &e^x \\
        &\frac{d}{dx}\ln(x)      &= &\frac{1}{x} \\
        &\frac{d}{dx}\log_b(x)   &= &\frac{1}{\ln(b) \cdot x} \\
        &\frac{d}{dx}\sin(x)     &= &\cos(x) \\
        &\frac{d}{dx}\cos(x)     &= &-\sin(x) \\
        &\frac{d}{dx}\tan(x)     &= &\frac{d}{dx}\frac{\sin(x)}{\cos(x)}  \\
         &&= &\frac{1}{\cos^2} \\
         &&= &1 + \tan^2(x) \\
        &\frac{d}{dx}\arcsin(x)  &= &\frac{1}{\sqrt{1-x^2}} \\
        &\frac{d}{dx}\arccos(x)  &= &-\frac{1}{\sqrt{1-x^2}} \\
        &\frac{d}{dx}\arctan(x)  &= &\frac{1}{1+x^2}
    \end{aligned}
\end{equation}
\begin{equation}
    \begin{aligned}
        \text{Linearität: }    \enspace    &\frac{d}{dx} (f(x)+g(x))           &= &f'(x)+g'(x) \\
        \text{Produkt: }       \enspace    &\frac{d}{dx} (f(x) \cdot g(x))     &= &f'(x)+g(x) + f(x)+g'(x) \\
        \text{Konstante: }     \enspace    &\frac{d}{dx} (c \cdot f(x))        &= &c \cdot f' \\
        \text{Kettenregel: }   \enspace    &\frac{d}{dx} (f(g(x)))             &= &f'(g(x))+g'(x) \\
        \text{Quotient: }      \enspace    &\frac{d}{dx} (\frac{f(x)}{g(x)})   &= &\frac{f'(x) \cdot g(x) - f(x) \cdot g'(x)}{(g(x))^2} \\
    \end{aligned}
\end{equation}

\section{Integration}
\section{Kurvendiskussion}
Stationäre Punkte bei \(f^{'}(x)=0 \), wobei:
\begin{enumerate}
    \item \( f^{''}(x)<0 \implies\) lokale Maxima
    \item \( f^{''}(x)>0 \implies\) lokale Minima
    \item \( f^{''}(x)=0 \implies\) Unentschieden
    \item \( f^{''}(x)=0 \land f^{'''}(x) \neq 0 \implies\) Wendestelle
\end{enumerate}
\section{Taylor}
\section{Fourier}
\section{Trigonometrische Funktionen}
Add Images here
\subsection{Spezielle Funktionswerte}
\begin{equation}
    \begin{aligned}
        \sin(\frac{\pi}{6}) = \frac{1}{2},
        \sin(\frac{\pi}{4}) = \frac{\sqrt{2}}{2},
        \sin(\frac{\pi}{3}) = \frac{\sqrt{3}}{2} \\
        \cos(\frac{\pi}{6}) = \frac{\sqrt{3}}{2},
        \cos(\frac{\pi}{4}) = \frac{\sqrt{2}}{2},
        \cos(\frac{\pi}{3}) = \frac{1}{2}
    \end{aligned}
\end{equation}
\subsection{Additionstheoreme und Produktformeln}
\begin{equation}
    \begin{aligned}
        \sin(a \pm b) &= &\sin(a) \cos(b) &\pm \cos(a) \sin(b) \\
        \cos(a \pm b) &= &\cos(a)\cos(b) &\mp \sin(a) \sin(b) \\
        \cos(a) \cos(b) &= &\frac{1}{2}(\cos(a-b) &+ \cos(a+b)) \\
        \sin(a) \sin(b) &= &\frac{1}{2}(\cos(a-b) &- \cos(a+b)) \\
        \cos(a) \sin(b) &= &\frac{1}{2}(\sin(a+b) &- \sin(a-b))
    \end{aligned}
\end{equation}
\subsubsection{Doppelwinkeltheorem}
\begin{equation}
    \begin{aligned}
        1 &= &\cos^2(x)+\sin^2(x) \\
        \cos(2x) &= &\cos^2(x)-\sin^2(x)\\
        \sin(2x) &= &2\sin(x)\cos(x)
    \end{aligned}
\end{equation}

\end{document}
