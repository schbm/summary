\documentclass[../Main.tex]{subfiles}

\begin{document}
\chapter{Statistical Machine Learning}

\intro{

}

\section{Basics}
\begin{figure}[H]
    \centering
    \includegraphics[width=1\linewidth]{Images/overview.png}
    \caption{Overview}
\end{figure}

\defn{Correlation and Causation}{
    \textbf{REMEMBER!} Correlation does not imply causality!
}

\defn{No Free Lunch}{
There is not one method which will 
beat all the other methods all the 
time. The better the assumptions of the 
method match the true state of nature, 
the better a particular method performs– This is sometimes referred to as the 
"No Free Lunch Theorem".
}

In StatML we want to create mathematical models based on data.
We assume a fixed but unknown relationship between the input variables \(X\) and our
output (dependent variable) \(Y\).

Given a dataset (usually represented by a input vector \(X\)):
\begin{equation}
    X = (X_1, X_2, \dots, X_p)
\end{equation}
Find a function that as good as possible models our unknown relationship:
\begin{equation}
    Y = f(X) + \varepsilon
\end{equation}
One of the main goals of statistical learning is to minimize the reducible error \(\varepsilon\).

\defn{Inference vs Prediction}{
    \begin{itemize}
        \item In \textbf{inference} we are interested in understanding the relationship between predictors and response.
        \item The goal of \textbf{prediction} is to calculate new values \(\hat{y}\) with given predictors \(X_i\).
    \end{itemize}
}

\begin{figure}[H]
    \centering
    \includegraphics[width=0.75\linewidth]{Images/regr-vs-class.png}
    \caption{Regression vs Prediction}
\end{figure}

\defn{Estimation Aproaches}{
\begin{description}
    \item[Parametric] Assumptions are made about the functional form of \(f(X)\), where some missing parameters need to be learned. For example a simple linear model assumes form: \(f(X)=\beta_0+\beta_1X_1\).  If the form of \(\hat{f}(X)\) is far from \(f(X)\), then this approach achieves badly.
    \item[Non-Parametric]  Methods that make no assumptions about the functional form of \(f(X)\).
\end{description}
}

When choosing a method or model. There is a clear tradeoff between flexibility and model interpretability.
One can say that a model which is easier to interprete, is in turn also less capable to predict accurate values, in comparison
to models which are less interpretable.

\begin{figure}[H]
    \centering
    \includegraphics[width=0.75\linewidth]{Images/model-interpretability.png}
    \caption{Model Interpretability Tradeoff}
\end{figure}

\defn{Supervised Learning}{
    In supervised learning for each input vector \(X\), its corresponding dependent variable \(Y\) is known.
}

\subsection{Error}
\defn{Error}{
    The Output (dependable variable) depends not only on the input, but also a from \(X\) 
    \textbf{indepentend zero mean error term} \(\epsilon\). Even a perfect prediction function \(f(x_i)\) has a 
    \textbf{irreducible error} \(\epsilon\).
    Error is introduced by internal as well as external factors:
    \begin{itemize}
        \item Random Measurement Error
        \item Systematic Errors e.g. humidity
    \end{itemize}

    If the estimated \(\hat{f}(X)\) is not equal to the true function, 
    there is the possibility to learn a better relationship between input and output, this is called \textbf{reducible error}. 
}

\exm{Error Terms with MSE}{
    One can show using algebra that the average error consists of two terms:
    \begin{equation}
        \begin{split}
            MSE&=E\{(Y-\hat{Y})^2\}\\
            &=(1)E\{(f(X)-\hat{f}(X))^2\}+(2)Var(\epsilon)\\
            (1)\text{Reducible Error }&+(2)\text{Irreducible Error}
        \end{split}
    \end{equation}
}

\newpage
\subsection{Bias-Variance}

\begin{figure}[H]
    \centering
    \includegraphics[width=0.75\linewidth]{Images/Bias_and_variance_contributing_to_total_error.svg.png}
    \caption{Bias Variance Tradeoff}
\end{figure}
The \textbf{expected test MSE} is the sum of a \textbf{variance} term, a \textbf{bias} term squared and the \textbf{irreducible error}.  Its defined as the average test MSE that would be obtained if \(f(X)\) is estimated repeatedly using a large number of training sets at tested at \(x_0\).
\begin{figure}[H]
    \centering
    \includegraphics[width=0.75\linewidth]{Images/bias-variance-test-mse.png}
    \caption{Test MSE Equivalent}
\end{figure}

The \(\sqrt{variance}\) refers to the amount by which the estimate would change,
if we estimated using a different training set. 
Ideally the training set should have little influence on how the method estimates, this is typical for \textbf{low flexibility} methods.

The bias refers to the error introduced by approximation using a model that is \textbf{too simple} (low flexibility)
to capture the \textbf{true shape} of a function, this will result in \textbf{high bias}.

\begin{figure}[H]
    \centering
    \includegraphics[width=0.5\linewidth]{Images/bias-variance-tradeoff.png}
    \caption{Bias Variance Tradeoff}
\end{figure}
\begin{figure}[H]
    \centering
    \includegraphics[width=0.75\linewidth]{Images/bias-variance-optimum.png}
    \caption{Bias Variance Tradeoff Optimum}
\end{figure}

\defn{Bias-Variance Summary}{
    In principle we say:
    \begin{itemize}
        \item \textbf{Flexible} methods have low bias but high variance
        \item \textbf{Simple} methods have low variance but high bias
    \end{itemize}
}
\begin{figure}[H]
    \centering
    \includegraphics[width=1\linewidth]{Images/bias-variance-tradeoff-overview.png}
    \caption{Bias Variance Overview}
\end{figure}

\defn{Model Selection Merksatz}{
    Don't select the method minimizing the \textbf{training MSE}!
}

\newpage
\section{Regression}
Regression is a method to calculate \textbf{quantitative} (numerical) predictions where \(Y\) is an ordered numerical value.
In its essence we define a mathematical model and an associated error function.
We then try to minimize the error, which is \textit{just} an optimization problem.

For a model to be optimizable many different well known optimization models exist.
For most regression problems we use \textbf{RSS}.
With the RSS we try to minimize the difference between squared sums of the prediction \(\hat{Y}\) and the actual data \(Y\).

The terms are squared to keep a positive sign. It can also be modeled with the abs function, 
but this makes it harder to find an optimization method due to it beeing harder to be optimized using analysis e.g. differentiation.
\defn{RSS}{
    \begin{equation}
        RSS = \sum (y - \hat{y})^2
    \end{equation}
}

\subsection{Model Accuracy}
We can model the accuracy of a model using the \textbf{MSE}.
In essence the MSE shows the on average commited error when predicting already known values.

\defn{Mean Squared Error}{
    \begin{equation}
        \begin{split}
            MSE &= \frac{1}{n}(y_i-\hat{f}(x_i))^2 \\
            MSE &= \frac{RSS}{n}
        \end{split}
    \end{equation}
}

\newpage
\subsection{Linear Regression}
In linear regression estimates are modeled with a linear relationship between a scalar response (dependent variable) 
and one or more explanatory variables (regressor or independent variable). There is a closed form solution for the RSS optimization.
\defn{Linear Regression Model}{
    We use a linear model (linear in \(b_i\)):
    \begin{equation}
        Y \approx \hat{Y} = \beta_0 + \sum_{i=0}^n \beta_{i+1} X_i
    \end{equation}
    \(b_0\) is also called a bias term.
    We find a optimal prediction function by minimizing the RSS:
    \begin{equation}
        RSS = e_1^2+\dots+e_n^2 \text{ where } e_i = y_i-\hat{y}_i
    \end{equation}
}

\defn{Coefficients in Simple Linear Regression}{
    \begin{equation}
        \begin{split}
            \hat{b}_1 &= \frac{\sum_{i=1}^{n} (x_i - \bar{x})(y_i - \bar{y}) }{\sum_{i=1}^{n} (x_i - \bar{x})^2} \\
            \hat{b}_0 &= \bar{y} - \hat{b}_1 \bar{x}
        \end{split}
    \end{equation}
}

\subsection{Linear Regression: Metrics}
\textbf{RSE} is roughly speaking, the average amount that the response will deviate from the true regression line. It provides an absolute measure of lack of fit of the model to the data. In fact the RSE carries the same unit as \(Y\).\textbf{ \(R^2\)} is the proportion of the variance explained by the model and hence it takes values between 0 and 1. \textbf{TSS} is the total sum of squares, which is the RSS if \(Y\) would always be predicted using the sample mean of \(Y\) (best possible predictor if we do not measure \(X\)). Assess accuracy of model with:
\defn{Model Accuracy}{
    \begin{equation}
        \begin{split}
            RSE &= \sqrt{\frac{1}{n-p-1}RSS} \\
            TSS &= \sum (y_i - \bar{y} )^2 = n \sigma_Y^2\\
            R^2 &= \frac{TSS-RSS}{TSS} = 1-\frac{RSS}{TSS} \\
        \end{split}
    \end{equation}
}

Since RSE includes the predictor count in the denominator it punishes adding more predictors to the model.
In TSS adding more predictors is always improving the metric.
\textbf{In simple linear regression} the following holds:
\begin{equation}
    R^2 = Cor(X,Y)^2 = Cor(Y,X)^2
\end{equation}
\textbf{In multivariate linear regression} the following holds:
\begin{equation}
    R^2 = Cor(Y,\hat{Y})^2
\end{equation}
We require a small variance of the estimate or equivalently a small standard error. Since \(\sigma^2\) is unknown it has to be estimated from data. Note that \(\sigma\) is the standard deviation of each independent realization of \(Y\).
\begin{equation}
    \begin{split}
         Var(\hat{\mu}) = SE(\hat{\mu})^2 = \frac{\sigma^2}{n}\\
         \text{where } \sigma^2 = Var(\epsilon)
    \end{split}
\end{equation}
Variance \(\sigma^2\) estimation using residual standard error (RSE):
\defn{Standard Error Estimation using RSE}{
    \begin{equation}
    \begin{split}
        \text{for }\sigma^2 &= Var(\epsilon) \\
        \implies SE\approx  \hat{SE} = RSE &= \sqrt{\frac{1}{n-p-1}RSS}
    \end{split}
\end{equation}
}

\subsection{Linear Regression: Predictor Correlation}
There is a common effect in practice, where in simple linear regression a predictor is significant (has a low p-value)
but is insignificant in multiple linear regression.
This can happen when there is a latent variable, which is correlated with two other variables.

When the other two variables are 
regressed onto each other, it appears 
that there is a relationship between 
them, even though the both depend 
on the same latent variable.

\defn{Correlation (Empirical)}{
    \begin{equation}
        Cor(X,Y) = \frac{\sum (x_i - \bar{x}) (y_i - \bar{y})}{\sqrt{\sum (x_i - \bar{x})^2} \sqrt{\sum (y_i - \bar{y})}}
    \end{equation}
    Here x and y does not denote dependent or independent variables but any two random variables e.g two predictors.
}

The correlation values are best interpretable in a normalized correlation matrix:
\begin{figure}[H]
    \centering
    \includegraphics[width=0.5\linewidth]{Images/corr-example.png}
    \caption{Normalized Coeff Correlation Example}
\end{figure}

\subsection{Linear Regression: Prediction Errors}
\defn{Errors in Prediction}{
    When predicting values we have three possible errors:
    \begin{enumerate}
        \item Inaccuracy in estimating the coefficients (Use confidence intervals)
        \item Model bias (Assume linear model fits good enough)
        \item The irreducible error \(\varepsilon\) (Prediction interval)
    \end{enumerate}
}

Confidence intervals quantify the uncertainty surrounding the averages of a prediction.
Here we are not interested in the overall \(\varepsilon\), since it does not contribute to the uncertainty
in the confidence interval. This is because \(\varepsilon\) is zero in the linear model.
If you can afford to average over many take confidence intervals otherwhise choose prediction invervals

\exm{Confidence vs Prediction Intervals}{
    \begin{itemize}
        \item Confidence interval = 95\% \textbf{of all data samples} lie in that interval.
        \item Prediction interval = with 95\% chance, the true value \(y\) of \textbf{one data sample} lies in that interval
    \end{itemize}
}

Under the Gaussian assumption confidence intervals can be computed using the estimated variance. 
\defn{Confidence Intervals}{
    \begin{equation}
    Pr(a<\hat{b}_i<b)=0.95 \implies \hat{\beta}_i \pm 2\cdot \hat{SE}(\hat{\beta}_i)
\end{equation}
}

\subsection{Predictor Selection}
\begin{itemize}
    \item Forward selection
    \item Backward selection
    \begin{itemize}
        \item Cannot be used if there are more predictor variables than training samples
    \end{itemize}
    \item Mixed Selection
\end{itemize}

\subsection{Assessing Coefficients: T-Test}
The T-test (with the T-statistic), is a tool for evaluating the means of one or two populations using hypothesis testing) Hypothesis tests using null hypothesis. We can test the relevance of coefficients using the T-test:
\defn{T-Test for Coefficients}{

\begin{equation}
    \begin{split}
        H_0 : \text{There is no relation between }X \text{ and }\\
        \text{or } b_i = 0 \\
        H_a : \text{There is a relationship}
    \end{split}
\end{equation}
\begin{equation}
    \text{T-statistic} = \frac{\hat{\beta}_i}{\hat{SE}(\hat{\beta}_i)}
\end{equation}
With \(n-n_\beta\) degrees of freedom where p:
\begin{equation}
    p = Pr(|T|>|t| \qquad | H_0)
\end{equation}
}
\textbf{P-Value} is the probability that we realistically observe an absolute T-value equal or bigger to the one observed under the \(H_0\).

\subsection{Assessing Coefficients: F-Test}
An F-test is any statistical test used to compare the variances of two samples or the ratio of variances between multiple samples. To check if there is any relationship between predictors and the dependent variable the F-test can be used:
\defn{F-Test}{
    \begin{equation}
    \begin{split}
        H_0 &: \beta_0 = \dots = \beta_n = 0 \\
        H_a &: \text{at least one } \beta_j \text{ is non-zero} \\
        \text{using F-statistic} &: \frac{\text{explained variance}/p}{\text{unexplained variance}/(n-p-1)} \\
        \text{using F-statistic} &: F=\frac{(TSS-RSS)/p}{RSS/(n-p-1)}\\
        \text{and} &: TSS = \sum (y_i-\bar{y})^2 = n\sigma^2 \text{, } RSS=\sum (y_i - \hat{y})^2
    \end{split}
\end{equation}
}
F-statistic expresses the improvement of the model per parameter \(p\) in multiples of the residual variance. If the linear model assumptions are correct and \(H_a\) is true we expect the F-statistic on average to be greater than one.

One can also use a subset of \(q\) coefficients to test with a new hypothesis. We order the predictor variables so that the last q variables are the one to test for being zero. Then we fit a second model that uses all the predictor variables except the last q .
\begin{equation}
    F = \frac{(RSS_0-RSS)/q}{RSS/(n-p-1)}
\end{equation}
\defn{Single Coefficient F-Test}{
    When we leave out only one variable (q=1) then the F-statistic would tell the partial effect of adding that variable to the model. \textbf{It turns out, the F-statistic, if only one variable is left out, is identical to the squared T-statistic.}
}

\begin{figure}[H]
    \centering
    \includegraphics[width=0.75\linewidth]{Images/f-dist.png}
    \caption{F-Distribution}
\end{figure}

\subsection{Linear Regression: Potential Problems}
\begin{enumerate}
    \item Non-linearity of data -> look at residual plots -> transform inputs
    \item Correlation of error terms -> caution when collecting data!
    \item Non-constant variance of error terms -> look at residual plots -> transform output
    \item Outliers (unusual output value) -> if \(\frac{\hat{\epsilon}_i}{\hat{\sigma}\sqrt{1-h_i}}>3\) -> outlier
    \item High-leverage points (unusual input value) -> if \(h_i = \frac{1}{n} + \frac{(x_i - \bar{x})^2}{\sum_{i=1}^{n} (x_i - \bar{x})^2} > (p+1)/n\) -> h.l.p
    \item Collinearity -> if \(VIF(\hat{b}_j) = \frac{1}{1-R_{X_j|X_{-j}^2}}>5\) -> problem
\end{enumerate}

\subsection{Qualitative Predictors}
Binary qualitative predictors can be modeled in linear regression using a technique that maps
the class to either 0 or 1 this is often called dummy or indicator variables:

\begin{equation}
    x_i = \begin{cases}
        1 & \text{if class x}\\
        0 & \text{if class y}
    \end{cases}
\end{equation}
The model can then be extended to:
\begin{equation}
    y_i = b_0 + b_1 x_1 + \varepsilon_i = \begin{cases}
        b_0 + b_1 + \varepsilon_i \\
        b_0 + \varepsilon_i 
    \end{cases}
\end{equation}
This simplifies into a linear function with slope influence or only with a bias term.

\begin{figure}[H]
    \centering
    \includegraphics[width=0.75\linewidth]{Images/indicator-function.png}
    \caption{Dummy/Indicator Variable}
\end{figure}

If multiple qualitative predictors are to be modeled new dummy variables need to be introduced.
There always only needs to be \(l-1\) dummy variables where l is the number of levels. The level
without a dedicated dummy variable is called the baseline.

It is better to use the F-test to judge if the predictors are relevante,
since the individual p-values of the coefficients rely on the chosen encoding of the dummy variable. 

\subsection{Removing the additive assumption (Interaction Terms)}
The linear model assumes that the relationship between the predictors and the response are additive and linear.
Additivity implies that the change in the response due to a unit change in a predictor is independent of the value
of the other predictors.

\defn{Cross-Terms}{
    We can remove additivity by introducting cross-terms:
    \begin{equation}
        Y = b_0 + b_1 X_1 + b_2 X_2 + b_3 X_1 X_2 + \varepsilon
    \end{equation}
}

The hierarchical principle states, that if there is an interaction 
term in a model, then the two main effects should also be 
included even when their p-values are large.

\subsection{Non-linear relationships TBD}

\newpage
\section{Classification}
In classification we make qualitative (categorical) predictions where \(Y\) is a discrete categorical value.
As in regression there are many methods to quantify the error.

\subsection{Model Accuracy}
One of the most general methods is to model error with the Classification Error Rate (CER):
\defn{Classification Error Rate}{
\begin{equation}
    Error = \frac{1}{n}\sum_{i=1}^n I(y_i \ne \hat{y}_i)
\end{equation}
}

\subsection{Model Accuracy with Confusion Matrix}
\begin{figure}[H]
    \centering
    \includegraphics[width=0.75\linewidth]{Images/conf-matrix.jpg}
    \caption{Confusion Matrix}
\end{figure}

\subsection{Threshold selection using ROC}
receiver operating characteristic curve, or ROC curve,
is a graphical plot that illustrates the performance of a binary classifier model
(can be used for multi class classification as well) at varying threshold values.

\begin{figure}[H]
    \centering
    \includegraphics[width=0.75\linewidth]{Images/roc.png}
    \caption{ROC Graph}
\end{figure}

It is a graph that plots the true 
positive rate (TP/P) as a function of 
the false positive rate (FP/N).
Each binary classification scheme has 
a ROC, which is independent of the 
selected threshold.
The optimal curve hugs the left upper 
corner and is non-decreasing.
Hence the \textbf{area under the curve (AUC)} is a 
good measure for the fundamental 
performance of a binary classification 
scheme.

\defn{Important Terms}{
    \begin{description}
        \item[Sensitivity] can you find, what you want
        \item[Specificity] can you find, what you don't want
    \end{description}
    Ideally both, sensitivity and specificity are 
    high. Sensitivity is also called Recall.
}
\subsection{Bayes-Classifier}
A Bayes classifier results in the lowest possible overall test error rate:
\defn{Bayes Classifier}{
    Test error rate is minimized, by a classifier that assigns each observation to the most probable class, given its predictor value. Results in the lowest possible error rate called Bayes error rate.
    \begin{equation}
        \begin{split}
            &Pr(Y=j|X =x_0)\\
            \text{set class to: }max_j(&Pr(Y=j|X=x_0))
        \end{split}
    \end{equation}
    Where \(j\) is the most probable class for a given input variable (predictor) \(X=x_0\). E.g for a 2 class model: \(Pr(Y=1|X=x_0)>0.5\)

    The bayes error rate is analogous to the irreducible error and is 0.1304.
}

\newpage
\subsection{KNN}
\defn{K-Nearest Neighbors}{
    Bayes classifier requires knowledge of the conditional distribution of \(Y|X\), in a real world problem this is never known. K-nearest neighbors is the simplest of such methods in which given a test point \(x_0\) KNN finds \(K\) neighbors and then estimates the class probabilities as the fraction of neighbors which belong to a particular class.

    \begin{equation}
        Pr(Y=j|X=x_0)=\frac{1}{K}\sum_{i\in N_0} I(y_i=j)
    \end{equation}
    Where \(N_0\) are the nearest neighbors to the point \(x_o)\).
    \begin{itemize}
        \item Small K results in high variance
        \item Big K results in high bias
        \item K controls the trade off
        \item \(\frac{1}{K}\) is a measurement of flexibility
    \end{itemize}
    
}
\begin{figure}[H]
    \centering
    \includegraphics[width=0.75\linewidth]{Images/k-nearest.png}
    \caption{KNN}
\end{figure}

\begin{figure}[H]
    \centering
    \includegraphics[width=0.5\linewidth]{Images/knn-flexibility.png}
    \caption{KNN Flexibility}
\end{figure}

\newpage
\subsection{Logistic Regression}
Linear regression does not work well for classification.
We instead coul use logistic regression where a probability [0;1] is modeled.

 \begin{equation}
    log(\frac{p(X)}{1-p(X)}) = b_0 + b_1 X_1 + \dots + b_p X_p
 \end{equation}

 \defn{Logistic Regression}{
    \begin{equation}
        p(X) = \frac{e^{b_0 + b_1 X_1 + \dots + b_p X_p}}{1+ e^{b_0 + b_1 X_1 + \dots + b_p X_p}}
     \end{equation}
 }

 The parameters are found with the Maximum Likelihood approach, where we maximize:
 \begin{equation}
    l(b_0,b_1) = \prod_{i:y_i=1} p(x_i) \prod_{i':y_{i'}=0} (1-p(x_{i'}))
 \end{equation}

 \begin{figure}[H]
    \centering
    \includegraphics[width=0.75\linewidth]{Images/lin-reg-vs-log.png}
    \caption{Linear vs Logistic Regression}
\end{figure}

If we had three categories, we would 
need to estimate the probabilities of 
each of them, given the observed 
predictors.

Problems with Logistic Regression:
\begin{itemize}
    \item For well-separated classes: unstable results
    \item For small n: unstable results
    \item Workaround required for multiple classes
\end{itemize}

\subsection{Comparison of Methods}
\begin{multicols}{2}
    \subsubsection{Logistic Regression vs LDA}
    \begin{itemize}
        \item They often have very similar 
        performance 
        \item LDA tends to outperform logistic 
        regression if the Gaussian 
        assumption is approximately correct
        \item Logistic regression tends to 
        outperform LDA if the Gaussian 
        assumption is basically wrong
    \end{itemize}
    \subsubsection{KNN}
    \begin{itemize}
        \item  This is a non-parametric scheme, 
        hence no model is assumed
        \item Hence one would expect KNN to 
        outperform LDA and logistic 
        regression if the decision boundaries 
        need to be non-linear
        \item But KNN does not tell us which predictors 
        are important and need lots of training data
        \item Does not work at all in high dimensional 
        settings 
    \end{itemize}
    \subsubsection{QDA}
    \begin{itemize}
        \item In some sense this is a compromise 
        between simple linear boundaries of 
        LDA/logistic regression and arbitrary 
        boundaries of KNN
        \item Here the boundaries are 
        hyperparabolas, which are much 
        more complicated than hyperplanes 
        but still much less flexible than KNN 
        boundaries
        \item Hence it does take more training data 
        than the linear methods but much 
        less than KNN to perform well
    \end{itemize}
\end{multicols}

\newpage
\section{Classification with Linear Discriminant Analysis}
\subsection{With p=1}
In logistic regression: we directly 
model the probability \(P(Y | X)\).

In linear discriminant analysis we go 
the other way around, we estimate 
the probability density functions 
(pdf's) of the observations \(X\), given a 
particular class Y: i.e., \(p(X|Y)\)


If we model these pdfs as Gaussian, we 
result in a scheme very similar to logistic 
regression.


Then we apply Bayes' theorem to flip 
these probabilities around and get 
the ones on the top right, which 
allow us to make an optimal decision.

\begin{equation}
    P(Y|X) = \frac{P(X|Y)P(Y)}{P(X)}
\end{equation}

This is done by introducing the following new variables:
\begin{equation*}
    \begin{split}
        \pi_k &\quad \text{prior probability for class }k\\
        f_k(X) &\quad \text{pdf of } X \text{ for a class } k
    \end{split}
\end{equation*}
We then insert them:
\begin{equation*}
    P(Y=k|X=x) = \frac{\pi_k f_k(x)}{\sum_{l=1}^{K} \pi_l f_l(x)}
\end{equation*}

Estimating the prior probabilities \(\pi_k\) is 
straight forward. Simply use the fraction of samples in the 
training set that belong to class \(k\) as the 
estimate of the prior probabilities:
\begin{equation*}
    \pi_k = \frac{n_k}{n}
\end{equation*}

Estimating \(f_k(X)\) is more challenging.
In absence of any additional information, we use a Gaussian with different mean \(\mu_k\) and same variance \(\sigma_k^2\):
\begin{equation*}
    f_k(x) = \frac{1}{\sqrt{2\pi}\sigma_k} exp(-\frac{1}{2\sigma_k^2} (x - \mu_k)^2)
\end{equation*}

We further simplify the model by 
assuming that all \(k\) classes have the 
\textbf{same variance}. Hence:
\begin{equation*}
    \sigma_k^2 = \sigma^2
\end{equation*}

We can simplify this expression \(P(Y=k|X)\) to \(\varphi_k(x)\):
\begin{equation}
    P_k(x) = x \frac{\mu_k}{\sigma^2} - \frac{\mu_k^2}{2\sigma^2} + log(\mu_k)
\end{equation}

For two classes we could construct a new decision boundary:
\begin{equation*}
    f_b(x) = P_1(x)-P_2(x)
\end{equation*}
This could indicate class 1 for values greater than zero, class 2 for values smaller than zero and
the boundary line where the probabilities are equal.
This can be reformed to the decision boundary of:
\begin{equation*}
    x = \frac{1}{2} (\mu_1+\mu_2)
\end{equation*}
If the observation x falls exactly between the 
means of the two classes, then it could 
belong equally well to either class.

\begin{figure}[H]
    \centering
    \includegraphics[width=0.75\linewidth]{Images/lda-decision-boundary.png}
    \caption{LDA Decision Boundary}
\end{figure}

\defn{LDA with p=1}{
The LDA method approximates the 
Bayes classifier by assuming Gaussian 
pdfs and using the following 
estimates for the means and the 
common variance.

\begin{equation}
    \begin{split}
        \hat{\mu}_k &= \frac{1}{n_k} \sum_{i:y_i=k} x_i \\
        \hat{\sigma}^2 &= \frac{1}{n-K} \sum_{k=1}^{K} \sum_{i:y_i=k} (x_i - \hat{\mu}_k)^2 \\
        \hat{\pi}_k &= \frac{n_k}{n} \\
        P_k(x) &= x \frac{\mu_k}{\sigma^2} - \frac{\mu_k^2}{2\sigma^2} + log(\mu_k)
    \end{split}
\end{equation}

Pick theclass with the largest \(P_k\)
}

\newpage
\subsection{Multiple Predictors p>1}
Extending LDA to multiple predictors 
results in the need for a model of a 
multidimensional pdf.

We will select a multidimensional 
Gaussian pdf where the means of the 
classes are different but all classes 
have the same covariance matrix.

\defn{Covariance Matrix}{
    Variance is defined as:
    \begin{equation}
        var(X) = E\{(x-\mu)^2\}
    \end{equation}
    In or more formally:
    \begin{equation*}
        var(X) = E(X^2)-E(X)^2
    \end{equation*}
    In multidimensional cases, this is the covariance matrix:
    \begin{equation}
        \Sigma = E\{ (x- \mu )(x -\mu)^T \}
    \end{equation}

    Since the correlation coefficient \(p_{ij}\) is given by:
    \begin{equation}
        p_{ij} = \frac{\sigma_{ij}}{\sigma_i \sigma_j}
    \end{equation}
    The matrix can thus be written as e.g:
    \begin{equation}
        \Sigma = \sigma_1 \sigma_2 \begin{bmatrix}
            \frac{\sigma1}{\sigma_2} & p_{12} \\
            p_{12} & \frac{\sigma_2}{\sigma_1}
        \end{bmatrix}
    \end{equation}
    Here \(x\) is a vector of observations.
    \(\mu\) is the mean vector, where each class has its own mean vector \(\mu_k\).
    \(\Sigma\) is the \(pxp\) covariance matrix that is common to all the classes.
}

After some algebra, a similar 
discriminant formula results in the 
multidimensional LDA as in the one
dimensional one:
\defn{Multidimensional LDA}{
    \begin{equation}
        \begin{split}
            \hat{\mu}_k &= \frac{1}{n_k} \sum_{i:y_i=k} x_i \\
            \hat{\Sigma} &= \frac{1}{n-K} \sum_{k=1}^{K} \sum_{i:y_i=k} (x_i - \hat{\mu}_k) (x_i - \hat{\mu}_k)^T \\
             \hat{\pi}_k &= \frac{n_k}{n} \\
             P_k(x) = x^T \Sigma^{-1} \mu_k - \frac{1}{2} \mu_k^T \Sigma^{-1} \mu_k + \log \pi_k
        \end{split}
    \end{equation}

    With little training data use LDA, otherwise QDA.
}

\begin{figure}[H]
    \centering
    \includegraphics[width=0.75\linewidth]{Images/lda-dec-boundary.png}
    \caption{LDA Decision Boundary}
\end{figure}

\begin{figure}[H]
    \centering
    \includegraphics[width=0.75\linewidth]{Images/lda-dec-boundary-2.png}
    \caption{LDA Decision Boundary}
\end{figure}

\subsection{Error Intepretation}
In LDA the goal is 
to approximate a Bayes classifier which is 
optimized for minimum overall error rate.
Minimizing the overall error rate is not always the desired goal.

We can vary the threshhold to vary the error rate. We can use ROC
to find an optimal threshold.

\newpage
\section{Classification with Quadratic Discriminant Analysis}

Quadratic discriminant analysis (QDA) 
allows for each class to have its own 
covariance matrix
The fundamental effect is, that the 
discriminant functions per class for a 
given observation x changes to 
include a new quadratic term.

\begin{figure}[H]
    \centering
    \includegraphics[width=0.75\linewidth]{Images/qda-vs-lda.png}
    \caption{QDA vs LDA}
\end{figure}

\defn{QDA}{
    Assumption:\( p(x|y=k)\) is Gaussian, where eachclass \(k\) has its
    own mean vector \(\mu_k\) and each class has its own covariance matrix \(\Sigma_k\).

    \begin{equation}
        P_k(x) = - \frac{1}{2} x^T \Sigma_k^{-1} (x - \mu_k) - \frac{1}{2} \log |\Sigma_k| + \log \pi_k
    \end{equation}

    QDA has more parameters than LDA thus increasing variance and decreasing bias.
    The QDA is quadratic in x since each class has its own covariance matrix hence the boundaries are hyperparapolas.
    With little training data use LDA, otherwise QDA.
}
\newpage

\section{Resampling Methods}
These are methods which use the given training 
set over and over again, this is achieved by sampling that set, i.e., 
create different subsets. To each of these subsets, the model 
can be fitted which allows for a statistical evaluation of the fitting.

\subsection{Cross-Validation}
Cross-validation can be used to estimate 
the test error in order to evaluate the 
performance of a method and finding an 
optimal flexibility setting (i.e., optimal hyperparameters).

\begin{figure}[H]
    \centering
    \includegraphics[width=0.25\linewidth]{Images/cross-validation.png}
    \caption{Cross-Validation}
\end{figure}

The basic idea is to hold out some samples from the 
training set and use these as a test set at the end, to 
estimate the final performance.

\begin{figure}[H]
    \centering
    \includegraphics[width=0.75\linewidth]{Images/cross-val-comparison.png}
    \caption{Cross-Validation Comparison}
\end{figure}

\begin{figure}[H]
    \centering
    \includegraphics[width=0.5\linewidth]{Images/different-cross-validation-methods.png}
    \caption{Different Cross Validation Methods}
\end{figure}

\begin{multicols}{2}

    \subsection{Validation Set Approach} 
    The given observations are randomly 
    split into a training set and a 
    validation set (hold-out set)

    The model is then trained using the 
    training set.
    After training the model is tested 
    using the validation set.

    \subsubsection{Pros}
    The validation set approach is 
    conceptually simple and easy to 
    implement, but there are two 
    potential problems

    \subsubsection{Cons}
    \begin{itemize}
        \item The validation MSE (an estimate of test 
        MSE) can vary by a large amount since it 
        depends on which observations are in the 
        training set and which observations are in 
        the validation set.
    
        \item Since only a subset if the observations is 
        used to train the model, the resulting 
        model is probably not as good as it could 
        be.
    \end{itemize}

    \subsection{Leave-one-out cross-validation (LOOCV)}
    LOOCV splits the observations into two parts.
    But they are now NOT of comparable size, 
    since only ONE observation is used for the validation set.
    Now the model can be fitted to almost all 
    data (n-1) and then the prediction is made 
    for the one single sample that was left out.


    The variance can be reduced by repeating 
    the procedure by systematically leaving out 
    one sample after the other and then 
    averaging the resulting single sample MSEs:
    \begin{equation*}
        CV_{(n)} = \frac{1}{n} \sum MSE_i
    \end{equation*}

    \subsubsection{Pros}
    \begin{itemize}
        \item (n-1) data points are used to fit the model.  Hence there is almost no bias.
        \item  No randomness in the procedure. Running LOOCV twice, will result in the 
        identical values for the estimated test MSE.
        \item We can use a trick where we only need to fit data once. (See Internet)
    \end{itemize}

    \subsubsection{Cons}
    \begin{itemize}
        \item LOOCV is computationally quite 
        expensive, since the model has to be 
        fitted n times.Usually, model fitting is slow, and n is large
    \end{itemize}

    \subsection{K-Fold}
    The basic idea is to split the 
    observations randomly in k
    equally sized sets.

    Then each set is left out while the other 
    sets are used to fit the model and the left 
    out set is used to calculate the particular 
    validation MSE.

    After k model fittings, the k validation MSEs 
    are average into an estimate of the test 
    MSE:
    \begin{equation*}
        CV_{(k)} = \frac{1}{k} \sum_{i=1}^{k} MSE_i
    \end{equation*}

    Clearly LOOCV is a special case of k
    fold cross-validation, where k=n.

    \subsubsection{Pros}
    k-fold cross validation improves on LOOC since it is not as computationally expensive.
\end{multicols}

\subsection{Test Error Rate in k-Fold}
\begin{equation}
    CV_{(k)} = \frac{1}{k} \sum_{i=1}^{k} Err_i
\end{equation}

The \textbf{training set} is then usually \textbf{split again} to create a 
validation set, for tuning hyperparameters.

\subsection{Bootstrap}
Most commonly used to provide a measure 
of accuracy of a parameter estimate and/or 
of a given statistical learning method.
Used extensively for tree based schemes.


The overall goal is to estimate uncertainty of an estimator.
The bootstrap method allows us to 
emulate the process of obtaining 
new sample sets

\begin{figure}[H]
    \centering
    \includegraphics[width=0.5\linewidth]{Images/bootstrapping.png}
    \caption{Bootstrapping}
\end{figure}

This is achieved by repeatedly sampling 
observations from the original data set
Hence, we can estimate the variability of 
an estimator without truly having new 
samples.


The sampling is performed with 
replacement, meaning the observation 
stays in the data set and can be selected (at 
random) again.

In general almost all machine 
learning schemes can be improved by 
building many models on 
bootstrapped data and then 
averaging the response of these 
scheme to a new input.

Computationally this is clearly an 
expensive way to improve 
performance.

\subsubsection{Concept 1} Use the B datasetsto
 estimate the uncertainty (e.g.
 standard error) of a method
\subsubsection{Concept 2} Train a separate ML model on each ofthe B datasets and 
average their outcomesto reduce variance

\begin{figure}[H]
    \centering
    \includegraphics[width=1\linewidth]{Images/bootstrapping-ensemble.png}
    \caption{Ensemble using Bootstrapping}
\end{figure}

\newpage
\section{Unsupervised Learning}
In UL there is no response variable \(Y\), only
predictors \(x_1,\dots ,x_n\). These are measured in
\(n\) observations resulting in the data matrix \(X\)
of dimension \(nxp\). The goal is to gather
inference data using methods like dimensionality reduction
and clustering. The methods require human judgement and have
thus a subjective nature.
\subsection{Principal Component Analysis}
You have your dataset \(X\) up to \(X_p\) with \(n\) rows.
Now you reduce it to include fewer features by incorporating
information into less variables. In PCA regression the idea
was to reduce the dimensions and then to apply a regression model.
This will lead to lower flexibility and thus lower variance.

Without regression we skip the training of a prediction model.
The method stays the same. We try to fit a line to our data
that contains the largest variance.

\defn{PCA}{
    Find a linear transform for principal component for which
    a set of features is a normalized linear combination
    of the features, resulting in the largest variance:
    \begin{equation}
        \begin{split}
            Z_1 &= \Phi_{11} X_1 + \Phi_{21}X_2+\dots+\phi_{p1}X_p \\
            \text{For which: } Z_1 = &max(var(Z_1))
        \end{split}
    \end{equation}
    \(Z_1\) denotes the first principal component. We choose
    a linear mapping to keep a simple representation.
    One could also choose a non linear transformation (T-SNE).

    With the loading vector of \(Z_1\):
    \begin{equation}
        \Phi_1 = (\Phi_{11},\dots,\Phi_{p1})^T
    \end{equation}
    Each entry \(\Phi\) is called loadings of \(Z\).
    The sum of the loadings squared and summed up must equal to 1:
    \begin{equation}
        \sum_{j=1}^{p} \Phi_{j1}^2=1
    \end{equation}
    In other words the loadings vector must have norm or length of one.
    Otherwise, the variance can be made arbitrarily large, by selecting
    large loadings.

    Before applying PCA the first step would be to normalize the data
    matrix \(X\). This is done by centering the columns (predictor variables)
    to have zero mean, by subtracting the sample mean of each column from
    the respective column. Often also division by sample SE.

}

\subsubsection{Scatter-Plot}
PCA is very useful for data visualisation. For example, we
can use PCA to reduce dimension in order to create a scatter plot.
Without PCA the amount of plots would equate to:
\begin{equation}
    \binom{p}{2} = \frac{p(p-1)}{2}
\end{equation}



\subsection{Clustering}

\end{document}