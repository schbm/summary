\documentclass[../Main.tex]{subfiles}
\begin{document}
\chapter{Lagrange}

\intro{

}

\section{Cookbook}
\begin{description}
    \item[Independent] When you fix all but one coordinate, still have a continous range of movement in the free coordinate
    \item[Complete] Capable of locating all parts at all times
    \item[Holomomic] \# Dofs = \# coordinates needed to describe motion
\end{description}

\begin{equation}
    \begin{split}
        L &= T - U \\
        T &= \text{Kinematic Energy} \\
        U &= \text{Potential Energy} \\
        \Rightarrow &\frac{\partial L}{\partial q} - \frac{d}{dt} \frac{\partial L}{d \dot{q}} = 0 \\
        \Rightarrow &\frac{d}{dt} \frac{\partial L}{d \dot{q}_j} - \frac{\partial L}{\partial q_j} = Q_j
    \end{split}
\end{equation}
Where \(L\) is the lagrangian with \(q_j\) generalized coordinates and \(Q_j\) generalized forces.
Where as \(Q_j\):
\begin{equation}
    \sum F_i \cdot d r(\delta_j)
\end{equation}
Where \(F_i\) is one of the forces dotted (dot product, both are vectors) with dr (eg. delta x) which is movement.
The virtual displacements in all possible degrees of freedom.

\begin{enumerate}
    \item Determine degrees of freedome
    \item Choose \(q_i\) (coordinates)
    \item Verify; Complete, independent, holonomic
    \item Compute T + U
    \item Compute derivatives for each coordinate \(q_i\) 
\end{enumerate}

With external (non conservative) forces:
\begin{enumerate}
    \item For each \(q_j\) (generalized forces) find the generalized force
    \(Q_j\) that goes with it
    \item Compute the virtual work done \(\delta_w\) associated with
    the virtual displacement \(\delta_{qj}\)
    \item  \(\delta_{wj} = Q_j \delta_j\)
\end{enumerate}

\end{document}
